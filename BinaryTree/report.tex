\documentclass[twoside,a4paper,10pt]{article}
\usepackage{geometry}
\geometry{margin=1.5cm, vmargin={0pt,1cm}}
\usepackage{geometry}
\setlength{\topmargin}{-1cm}
\setlength{\paperheight}{29.7cm}
\setlength{\textheight}{25.3cm}

\usepackage{verbatim}
\usepackage{ctex}
\usepackage{amsfonts}
\usepackage{amsmath}
\usepackage{amssymb}
\usepackage{amsthm}
\usepackage{enumerate}
\usepackage{graphicx}
\usepackage{booktabs} 
\usepackage{fancyhdr}
\usepackage{xcolor}
\usepackage{tikz}
\usepackage{listings}
\usepackage{xcolor}
\usepackage{graphicx}
\usepackage{hyperref}
% some common command
\newcommand{\norm}[1]{\lVert #1 \rVert}
\newcommand{\modulus}[1]{\lvert #1 \rvert}

\begin{document}

\pagestyle{fancy}
\fancyhead{}
\lhead{林修弘 3200300602}
\chead{Binart Tree}
\rhead{\today}


\section*{问题:BinaryTree的设计}
课本提供的代码中, 给出的头文件\texttt{BinarySearchTree.h} ,\texttt{AvlTree.h},\texttt{ RedBlackTree.h}和 \texttt{SplayTree.h}分别实现了对应的功能. 由于是教学代码, 每一个头文件都是独立的. 现在, 请根据它们各自的用途, 整理它们的逻辑关系, 重构全部代码, 并用继承关系予以表达.

\section{BinaryTree类}
BinaryTree 类是一个模板类,用于实现二叉树的基本结构。它定义了一个受保护的结构 BinaryNode 和一个指向 BinaryNode 的指针 root。此外,它还声明了一系列虚拟函数,以便在派生类中进行具体实现。

\subsection{BinaryNode}
可以看到BinarySearchTree和SplayTree的派生类中的节点均使用了结构体\texttt{BinaryNode}, AvlTree和RedBlackTree的节点需要对于\texttt{BinaryNode}中的成员进行了扩充,所以可以用继承的关系处理这两种节点。

\subsection{虚拟函数}
给出派生类的共同函数
\begin{verbatim}
virtual const Comparable& findMin() const = 0;
virtual const Comparable& findMax() const = 0;
virtual bool contains(const Comparable& x) const = 0;
virtual bool isEmpty() const { return root == nullptr; }
virtual void printTree(ostream& out = cout) const = 0;
virtual void makeEmpty() = 0;
virtual void insert(const Comparable& x) = 0;
virtual void insert(Comparable&& x) = 0;
virtual void remove(const Comparable& x) = 0;
\end{verbatim}

\section{实现}
给出\texttt{TestBinaryTree.cpp},代码包含原有的\texttt{TestBinarySearchTree.h} ,\texttt{TestAvlTree.h},\texttt{TestRedBlackTree.h}和 \texttt{TestSplayTree.h},只要能够完成运行并且"Finish"或者"Complete"即表该树运行成功。

\subsection{疑点}
按照原有的\texttt{TestSplayTree.h}测试SplayTree时,我发现代码无法成功运行,但是当我把\texttt{NUMS4=30000}改成\texttt{NUMS4=3000}就能成功运行了。

\end{document}




