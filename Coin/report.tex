\documentclass[twoside,a4paper,10pt]{article}
\usepackage{geometry}
\geometry{margin=1.5cm, vmargin={0pt,1cm}}
\usepackage{geometry}
\setlength{\topmargin}{-1cm}
\setlength{\paperheight}{29.7cm}
\setlength{\textheight}{25.3cm}

\usepackage{verbatim}
\usepackage{ctex}
\usepackage{amsfonts}
\usepackage{amsmath}
\usepackage{amssymb}
\usepackage{amsthm}
\usepackage{enumerate}
\usepackage{graphicx}
\usepackage{booktabs} 
\usepackage{fancyhdr}
\usepackage{xcolor}
\usepackage{tikz}
\usepackage{listings}
\usepackage{xcolor}
\usepackage{graphicx}
\usepackage{hyperref}
% some common command
\newcommand{\norm}[1]{\lVert #1 \rVert}
\newcommand{\modulus}[1]{\lvert #1 \rvert}

\begin{document}

\pagestyle{fancy}
\fancyhead{}
\lhead{林修弘 3200300602}
\chead{动态规划}
\rhead{\today}

\section{测试结果}

以下是针对不同目标金额的测试结果:

\begin{table}[h]
\centering
\begin{tabular}{|c|c|c|}
\hline
目标金额 & 计算结果 & 运行时间 (秒) \\
\hline
11 & 2 & $2.31 \times 10^{-5}$ \\
35 & 2 & $3.19 \times 10^{-5}$ \\
40 & 3 & $3.21 \times 10^{-5}$ \\
30 & 2 & $2.54 \times 10^{-5}$ \\
20 & 2 & $1.93 \times 10^{-5}$ \\
52 & 4 & $4.00 \times 10^{-5}$ \\
157 & 9 & $5.66 \times 10^{-5}$ \\
104 & 8 & $3.26 \times 10^{-5}$ \\
313 & 16 & $9.20 \times 10^{-5}$ \\
112 & 7 & $3.50 \times 10^{-5}$ \\
\hline
\end{tabular}
\caption{测试结果}
\end{table}

\section{理论分析}

理论上,找零钱问题的时间复杂度为 $O(N \cdot M)$,其中 $N$ 是目标金额,$M$ 是硬币面额的数量。在此情况下,$N$ 的最大值是 313,$M$ 的值是 4,因此理论上的时间复杂度非常低,远低于实际观察到的运行时间。

\section{结论}

通过实际运行测试,我们观察到算法的运行时间非常短暂,通常在 $10^{-5}$ 秒的数量级。这证明了该算法在解决找零钱问题时的高效性,与理论分析结果一致。

\end{document}


