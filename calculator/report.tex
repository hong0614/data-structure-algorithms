\documentclass[twoside,a4paper,10pt]{article}
\usepackage{geometry}
\geometry{margin=1.5cm, vmargin={0pt,1cm}}
\setlength{\topmargin}{-1cm}
\setlength{\paperheight}{29.7cm}
\setlength{\textheight}{25.3cm}

\usepackage{verbatim}
\usepackage{ctex}
\usepackage{amsfonts}
\usepackage{amsmath}
\usepackage{amssymb}
\usepackage{amsthm}
\usepackage{enumerate}
\usepackage{graphicx}
\usepackage{booktabs} 
\usepackage{fancyhdr}
\usepackage{xcolor}
\usepackage{tikz}
\usepackage{listings}
\usepackage{xcolor}
\usepackage{graphicx}
\usepackage{hyperref}
% some common command
\newcommand{\norm}[1]{\lVert #1 \rVert}
\newcommand{\modulus}[1]{\lvert #1 \rvert}

\begin{document}

\pagestyle{fancy}
\fancyhead{}
\lhead{林修弘 3200300602}
\chead{计算器项目设计与测试说明 }
\rhead{\today}


\section{项目设计思路}
本项目实现了一个简单的计算器,能够从输入文件中读取表达式,将中缀表达式转换为后缀表达式,然后计算后缀表达式的值,并将结果写入输出文件。以下是项目的设计思路:

\begin{enumerate}
\item \textbf{Calculator类}: 这个类包含了计算器的主要功能。它有以下方法和属性:
    \begin{itemize}
    \item \texttt{Calculator(const std::string\& inputFileName, const std::string\& outputFileName)}:构造函数,用于初始化输入和输出文件流,同时检查文件是否成功打开。
    \item \texttt{int getOperatorPriority(char c)}:根据运算符的优先级返回一个整数。
    \item \texttt{bool isOperator(char c)}:检查一个字符是否为运算符。
    \item \texttt{std::string remove\_illegalchar(std::string expression)}:删除表达式中的非法字符。
    \item \texttt{std::queue<std::string> in\_to\_post(std::string expression)}:将中缀表达式转换为后缀表达式。
    \item \texttt{double cal\_post(std::queue<std::string> postfix)}:计算后缀表达式的值。
    \item \texttt{void processInputFile()}:读取输入文件,处理每行表达式,计算结果并写入输出文件。
    \item \texttt{std::ifstream inputFile}:用于读取输入文件的文件流。
    \item \texttt{std::ofstream outputFile}:用于写入输出文件的文件流。
    \end{itemize}
\end{enumerate}

\section{测试说明}
为了测试计算器的功能,我们提供了一个输入文件\texttt{input.txt},其中包含了多个表达式,每行一个。以下是输入文件的内容:

\begin{lstlisting}
3.1 * (1.2 + 3.5)
1.2sdf3!!3 - 0.23
1t.y21 * (3RR4T + 1s.s00)
1+)2
1((2
4/0
\end{lstlisting}


\subsection{运行程序}
为了运行程序,只需执行\texttt{main()} 函数即可。程序将从\texttt{input.txt} 读取表达式,并将结果写入\texttt{output.txt} 文件。

\begin{lstlisting}
int main() 
{
    Calculator calculator("input.txt", "output.txt");
    calculator.processInputFile();
    return 0;
}
\end{lstlisting}

\subsection{结果分析和结论}
根据输入文件中的表达式,程序将计算结果写入输出文件。结果如下:

\begin{tabular}{|c|c|}
\hline
\textbf{输入表达式} & \textbf{测试结果} \\
\hline
3.1 * (1.2 + 3.5) & 14.57 \\
\hline
1.2sdf3!!3 - 0.23 & 1.003 \\
\hline
1t.y21 * (3RR4T + 1s.s00) & 42.35 \\
\hline
1+)2 & Error: Unmatched closing parenthesis. \\
\hline
1((2 & Error: Unmatched opening parenthesis. \\
\hline
4/0 & Error: Division by zero is not allowed. \\
\hline
\end{tabular}
\\
计算器成功计算了合法表达式,正确报告了括号不匹配的错误,以及不允许除以零的错误。因此,计算器在处理各种情况时表现出良好的稳定性和健壮性。

\end{document}


